\documentclass{article}

\title{NP, NPC, NPH}
\author{Dominik Lau}

\usepackage{blindtext}
\usepackage{amsmath}
\usepackage[utf8]{inputenc}
\usepackage[polish]{babel}
\usepackage[T1]{fontenc}
\usepackage{listings}
\usepackage{color}
\usepackage{amssymb}
\usepackage{esvect}
\usepackage{graphicx}


\graphicspath{ {./obrazy/} }

\definecolor{dkgreen}{rgb}{0,0.6,0}
\definecolor{gray}{rgb}{0.5,0.5,0.5}
\definecolor{mauve}{rgb}{0.58,0,0.82}

\lstset{frame=tb,
  language=Python,
  aboveskip=3mm,
  belowskip=3mm,
  showstringspaces=false,
  columns=flexible,
  basicstyle={\small\ttfamily},
  numbers=none,
  numberstyle=\tiny\color{gray},
  keywordstyle=\color{blue},
  commentstyle=\color{dkgreen},
  stringstyle=\color{mauve},
  breaklines=true,
  breakatwhitespace=true,
  tabsize=3
}


\begin{document}

\maketitle

\section{NP}
\subsection{Definicja}
Decyzyjny problem $\Pi \in$ NP $\iff$ jest rozwiązywalny w czasie wielomianowym przy zastosowaniu idei wyroczni weryfikowalne w czasie wielomianowym. \\\\
Decyzyjny problem $\Pi \in$ P $\iff$ jest rozwiązywalny w czasie wielomianowym. \\\\
Decyzyjny problem $\Pi \in$ NPI $\iff \Pi \in$ NP - P (NP-intermediate).  Problemy, dla których nie udowodniono, że są ani P ani NPC.\\\\
\textbf{Uwaga}
$\Pi \in$ P $\rightarrow \Pi \in$ NP 

\subsection{Algorytmy niedeterministyczne}
Algorytm wykonywany na niedeterministycznej maszynie Turinga, definiujemy działanie wyboru O(1)
zwracające dobry wynik dla zbioru danych.

\subsection{Przykładowy problem NPI}
Izomorfizm grafu

\subsection{$\alpha$-redukcja}
$\Pi_1 \alpha \Pi_2 \iff$  mamy funkcję T(x), która zachowuje problem i 
zmienia dane wejściowe $\Pi_1$ do $\Pi_2$. \\\\
\textbf{Istotne jest}, że $tr(\Pi_1) \leq tr(\Pi_2)$, gdzie $tr$ - trudność problemu.

\section{NPC}
\subsection{Definicja}
Decyzyjny problem $\Pi \in$ NPC $\iff$ $\Pi \in NP$ i $\forall_{\Pi_1 \in NP} \Pi_1 \alpha \Pi$. Czyli jest to problem
przynajmniej tak samo trudny jak wszystkie inne problemy w NP.

\subsection{3SAT i 3CNF}
\textbf{3CNF} to formuła logiczna składająca się z iloczynu klauzul, w których występują po trzy literały.
np. $\phi = (x_1 + x_2 + x_3)(\overline{x}_1 + x_4 + x_5)$ \\\\
\textbf{3SAT} to problem o pytaniu: Czy podana formuła $\phi$ 3CNF jest spełnialna tj. czy dla pewnego wartościowania zmiennych $\phi$, $\phi = 1$. Jest to jedyny problem NPC,  dla którego udowodniono bezpośrednio, 
że jest NPC (Cook,1971). Na chłopski rozum dlaczego tak jest: każdy algorytm można sprowadzić do
układu funkcji logicznych (np. układu bramek logicznych).
 
\subsection{Przykładowe problemy NPC}
\textbf{Pokrycie wierzchołkowe} \\\\
\textbf{3-wymiarowe skojarzenie} \\\\
\textbf{2-podział} \\\\
\textbf{Suma podzbioru} \\\\
\textbf{Genus grafu}

\section{NPH}

\end{document}
